\section{Copyright}

As far as IPRs that encourage competition and avoid monopoly go, appropriately used copyright is hard to beat. As has been repeatedly stated, copyright protects an expression of an idea, not an idea itself. When applied to software, this roughly means that a developer has no rights covering the exact function of the software, but can protect its source and object code. Were I to steal the source code for Microsoft Word, make a few trivial modifications, and sell the resulting program, I would have committed a copyright offence. If, however, I were to build my own program which allowed its user to write and format a document, Microsoft would be unable to touch me. This legal framework grants rightsholders control over where their software is distributed, how much people are charged to use it, and licensing agreements allow them to (in theory) control what a user is allowed to do with it.

Distributing software under open source is a common means for developers to grant their software the freedom to be modified, a freedom which traditional copyright restricts. Rightsholders choose to waive certain rights, and in return enthusiast programmers are encouraged to fix bugs, add features, and help guide the software's development. Open source alternatives exist for just about any conceivable piece of software, causing large software corporations to be in legitimate competition with hobbyists and enthusiasts. Initially, the spirit of open source software would appear to be one of cooperation rather than competition. It is tempting to suggest that, if all software were open source, all enthusiast developers would work on the same projects, and not split off to produce their own alternatives and compete with each other. This does not appear to be the case in software for either consumers or businesses, considering the fierce competition between Google Chromium and Mozilla Firefox, between OpenOffice and LibreOffice or between all the varying flavours of Linux. All of these pieces of software are open source, and are solidly competing %cite some browser and office usage stats
in a marketplace against myriad closed source alternatives. Indeed, Valve Software's recent foray into games consoles has presented a viable, open alternative to the more closed gaming platforms in a market which, for fourteen years, had been dominated by exactly three competitors. By protecting only the specific expression of the idea, copyright in all its forms enables and encourages the competitors within these marketplaces.

Whether the provided measure of protection is balanced is, however, another matter. Copyright is usually applied to more artistic media such as books and films, rather than to tools%cite
. Were I to buy a book, I would know that I couldn't find the story therein in any other book, but I'd have no guarantee that the story would be fit for any given purpose. Software is often marketed to its users as a means to an end, and sale of software constitutes a transfer of a license to use copyrighted material. The contracts under which they are licensed are varied beasts indeed. Some %cite!
will warrant that the software achieves its stated goals, others specify that the software is provided ``as-is"%cite!
, with ``no warranty given about the quality, functionality, availability or performance" of the software. This, unlike all other means to an end that a consumer may purchase, leaves the consumer thoroughly unprotected in the event that software turns out to be ineffective or faulty. Software is not sold as, nor is it considered by the public to be, anything even approaching a literary work. 

Another fault is that copyright of literary works protects the work for the life of the last surviving author plus 70 years, after which it falls into the public domain. The application of this to software is strange, in that software barely lasts more than 7 years. While a book may remain relevant and desirable to read thousands of years after the author's death, old and completely redundant versions of Microsoft Windows remain protected by copyright, despite being younger than the author of this essay. Further complicating this is the fact that software tends to be developed by large teams -- the most recent version of Windows had at least 875 people working on it %http://blogs.msdn.com/b/b8/archive/2011/08/17/introducing-the-team.aspx
. Enforcing the public's right to have this work fall into their domain will be non-trivial.

Copyright tends to protect rightsholders somewhat more than it protects the consumers, so it is difficult to consider its protection balanced.


\section{Patent}