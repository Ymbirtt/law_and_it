\section{Definitions and Preliminaries}

Computer software refers to programs running on a computer -- sequences of instructions which the computer carries out to achieve a particular goal. A developer will almost always produce source code, specifying high-level behaviour of the program in some abstract but human-readable way, which is then compiled into object code which the computer can understand. A developer may also import various libraries into their program. A library may be distributed as either source code or object code, and provides a pre-written method of performing a given task, eg Matplotlib %cite
is a library which allows mathematical plots to be made by python programs. Developers can still write code to perform these tasks without using a library, though generally they have neither the time nor the inclination to do so.

Source code is generally protected under copyright as a literary work %cite
. Copyright covers the expression of a substantial idea in permanent form, rather than an idea itself. The traditional business model for a developer is to write source code, keep it secret, and license the object code to consumers for a fee, exercising copyright over both the object and source. Throughout this essay we will use the term ``copyright" to refer to this model specifically, because it differs importantly from the idea of open source software.

Open source software is software whose source code is distributed freely to the public. In theory, anyone may acquire this source code and compile it to produce working object code that they can run on their computer at no cost. Open source software is still copyrighted, however the intention is to use copyright law to give more freedoms to their software. Open source software licenses tend to have a viral component, in that anyone who builds anything using an open source library is required to license their own work under a similar license. Open source developers can still make money from their software in various ways, for example a developer may, for a fee, license their software to a company under closed source conditions, breaking the viral chain.

All copyright works are subject to the idea of fair dealing%http://www.legislation.gov.uk/ukpga/1988/48/section/28
. If a small quantity of a copyrighted work is copied for certain purposes -- criticm, review, news reporting, research and private study, etc -- then no copyright infringement can occur. Otherwise, any copying, public performance or adaptation of a work infringes copyright. For software, fair dealing is rarely applicable, but making illegal copies of software remains an issue.
%http://www.legislation.gov.uk/ukpga/1988/48/part/I/chapter/II/crossheading/the-acts-restricted-by-copyright

Software patents are difficult to discuss, especially in EU legislation where the European Patent Convention explicitly states%cite
 that programs for computers are not eligible for patent, except where the program in question is not patented as a computer program, rather as a solution to an industrial problem. A program which moves files from one virtual location on a disk to another is not eligible for patent, since in and of itself it doesn't achieve any real industrial goal. A method for spraying paint over the surface of a car in a particularly efficient way may be granted a patent, even if this method can only be realistically realised as a computer program running on the robots in the factory. If a patent is granted, the patentee must publicly disclose the details of their invention, and is then granted the right to prevent others from making, using, selling or distributing this invention.
%http://www.ipo.gov.uk/patentsact1977.pdf - section 60, page 47. No seriously

In the following sections, we will discuss each of these rights in turn.