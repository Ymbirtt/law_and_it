Software developers broadly protect their intellectual property by way of copyright%cite
 -- various juristictions regard software as a literary work, %cite
giving rightsholders a degree of protection and authority over exactly who gets to distribute their software and under what conditions. In this essay we will compare traditional copyright protection with two alternative Intellectual Property Rights (IPRs), namely open source development and patent. We will discuss how they each reflect the qualities given in the title, as well as any intricacies and pitfalls that may arise from these rights, with particular emphasis on the idea of \emph{balanced protection} -- the rightsholder should not be protected from everything. The question of which IPR is more appropriate touches on the very difficult question of what the provision of software constitutes -- sale of goods or provision of service -- which will also be briefly discussed. Finally, we will discuss how well various IPRs can be applied to the increasingly vast market of video game software. Most of this essay will concern itself with UK and EU law, though some sections will necessarily need to stray into US law. If not otherwise indicated, assume UK law.